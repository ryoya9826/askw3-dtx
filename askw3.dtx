% 
% ^^A askw3.dtx
%
% \CheckSum{2466}
% \changes{v0.0}{2017/01/22}{Making Start}
% \changes{v1.0}{2018/10/10}{First release}
% \changes{v1.1}{2019/10/31}{New function added! See update history}
% \changes{v1.11}{2020/01/06}{sectionmark option is renewed}
% \changes{v2.0}{2020/05/01}{Add makethm, surrounded theorem environment}
% \changes{v2.1}{2021/03/10}{Add namelabelOP}
% \changes{v2.11}{2021/05/14}{Bug fix}
%
% \title{askw3 package~v2.11}
% \author{Ryoya ANDO \thanks{twitter:@Reincarnatorsan, \protect\url{https://ryoya9826.github.io/}}}
% \date{2021/05/14}
% \maketitle
% \iffalse
%<*driver>
\documentclass[dvipdfmx]{jltxdoc}
\let\cmd\relax
\usepackage{askw3}
\usepackage{manfnt}					%急カーブ
\usepackage{spverbatim}				%\verb中に改行
\usepackage{tikz,tikz-cd}
\usepackage{metalogo}
\usepackage{newtxtext,newtxmath}
\usepackage{multicol}
\usepackage{url}
\makethm{defi}{定義}
\makethm{thm}[defi]{定理}
\RecordChanges
\GetFileInfo{askw3.sty}
\begin{document}
	\DocInput{askw3.dtx}
	\PrintChanges
\end{document}
%</driver>
% \fi
%
% \iffalse
%<*askw3>
%%% This is file `askw3.sty',
%%% Copyright  2017-2021 Ryoya Ando (@Reincarnatorsan)
%%% GitHub:	https://github.com/ryoya9826/askw3-dtx 
%</askw3>
%<askw>\NeedsTeXFormat{LaTeX2e}
%<askw>\ProvidesPackage{askw3}[2021/05/14 v2.11]
%\fi
% \newenvironment{titebend}{\par\dbend\vspace*{1zh}\par}{\par\hspace*{\fill}\dbend\vspace*{1zh}\par}
%
% \section{Licenses} 
% 
%  修正BSDライセンス(The BSD 2-Clause License)の下で配布される.
% \section{User Manual}
%\subsection{Update History}
%\subsubsection*{v0.0, 2017/01/22}
%開発開始.
%\subsubsection*{v1.0, 2018/10/10}
%公開.
%\subsubsection*{v1.1, 2019/10/31}
%環境追加.itemize系,eqv環境.
%\subsubsection*{v1.11, 2020/01/06}
%パッケージオプションを(ほとんど)廃止して\cs{askwoption}を使用することに.Bug fix.
%\subsubsection*{v2.0, 2020/05/01}
%\cs{makethm}を正式に追加.
%\subsubsection*{v2.1, 2021/03/10}
%tlarrayパッケージへの依存関係を解消.人名索引機能(\cs{namelabelOP})を追加.
%\subsubsection*{v2.11, 2021/05/14}
%hyperrefまわりのバグを修正.
%\subsection{Introduction}
% このパッケージは拡張された定理環境を提供し,またいくつかの簡単なマクロを提供するものです.本パッケージは内部で xkeyval, amsmath, amssymb, amsthm, ascmac, bxghost, iftex, etoolbox パッケージを読み込みます. v1.1ではtlarrayパッケージを要求していましたが, tlarrayパッケージの作者がDEPRECATEDにしたのに伴い本パッケージでも使用を取りやめましたので入手する必要はありません,また後述するオプション\quo{links}を有効にした場合,hyperrefパッケージ,場合によりpxjahyperパッケージを読み込みます(本オプションは廃止予定です).
%
% 他のパッケージを読み込む際の順序として,hyperrefパッケージはaskw3.styより先に読み込んでください.またnewtxmathパッケージ等の数式フォントに手を加えるパッケージを読み込む場合は本パッケージの後に読み込んでください.
%
%本パッケージはドキュメントクラスが(bx)jsarticleであるドキュメントでの使用を想定しています. bxjsarticle.clsを用いている場合は\spverb|dvi=dvipdfmx, ja=standard, japaram={units}|をオプション引数に設定して使用していると想定して設計しています(もちろんLua\TeX などのdviウェアを設定する必要のないエンジンで|dvi=dvipdfmx|指定を行う必要はありません).
%
%\subsection{Package option}
%
% パッケージオプションを説明します.
%\subsubsection{links}
%
%\textbf{本オプションは廃止予定です!本パッケージを読む込む前にhyperrefパッケージ,pxjahyperパッケージを自身で読み込ませることを推奨します.}パッケージオプション\quo{links}を指定すると,もしhyperrefパッケージが読み込まれているならば,本パッケージが提供する枠付き定理環境についてのhyperlinkを提供します.もしhyperrefパッケージを読み込んでいない場合はhydelinks,hyperfiguresオプションでhyperrefパッケージを読み込み,同様の処理を行います.またp-\TeX 系列が実行されている場合はpxjahyperを読み込みます(plautopatch パッケージとは衝突しないはずです).
%
%\subsection{Surrounded theorem environment}
%\macro{makethm}
% まず本パッケージが提供する定理環境について説明します.基本的にはamsthmパッケージによる定理環境と共存が可能なように設計してあります.本パッケージ独自の定理環境を使用するには,次の構文\cs{makethm}\marg{envname}\marg{labelname}により環境を作成してから用います(Preambleでのみ使用可).それによって定義される環境\cs{begin}|{envname}|により,枠に囲われた定理環境を提供します.例えば\cs{makethm}|{defi}{定義}|により定義される\cs{begin}|{defi}|と\cs{end}|{defi}|によって;
%\begin{defi}
% contents
%\end{defi}
% が提供されます.この環境は省略可能な引数をとり,ラベルを表示することができます.例えば\cs{begin}|{defi}|[LaTeX]とすると;  
%\begin{defi}[LaTeXの定理]
%	contents
%\end{defi}
%といったぐあいです.新規に環境を作成するときには,番号付けをすでに定義された環境(厳密にはカウンタ)に追従させるかどうか選ぶことができます.その書式はamsthmパッケージの\cs{newtheorem}命令と同様です.例えば\cs{makethm}|{thm}[defi]{定理}|により;
%\begin{thm}
%  contents
%\end{thm}
% を得ます.また通し番号を振らない場合は\cs{makethm}$\ast$|{thm}{定理}|のようにしてください.
%^^A \m@syu@punctで後置する約物を指定可能.調整中.
%
%\begin{titebend}
% 厳密には2番目の省略可能引数には,すでに定義された環境名ではなく定義されたカウンタ名(内部で\cs{c@count}により管理されるカウンタに対する|{count}|を指定します.\LaTeX における\cs{addtocounter}などと同じ指定方法です)を指定します.例えば\cs{makethm}|{envname}[footnote]{labelname}|で番号が\cs{footnote}に追従するようになります(それを希望する状況はないでしょうけれど).
%\end{titebend}
% 番号付けに関連するオプションとして,Preambleで\macro{askwoption}\cs{askwoption}|{thmnumber=3}|としておくと,番号付けがpart,section,定理環境の順番で並んで行われるようになります.また定理環境の番号はsectionが変わるごとにリセットされるようになります.例を見てみましょう(この文書ではpartを使用していないので便宜上1を出力させています).\let\origthedefi\thedefi \def\thedefi{\arabic{part}.\thesection.\arabic{defi}}\setcounter{part}{1}
%\begin{defi}
%	contents
%\end{defi}\let\thepart\origpart
%
% 同様に\cs{askwoption}|{thmnumber=2}|によりsection,定理環境の順で番号付けされます.また,これらのオプションの処理の関係で\cs{makethm}コマンドはプリアンブルで実行するようにしてください.v2.1現在はthmnumber は1,2,3のみ指定可能です(1はデフォルトと等価です).注意点として,この連番機能はamsthmパッケージにより提供される\cs{newtheorem}命令によって作成された環境にも適用されることに注意してください.この機能はamsthmパッケージの拡張だと思ってしまうのがよいかもしれません.
%
%\cs{makethm}命令と\cs{newtheorem}命令のどちらで作成した環境も相互参照機能に対応しています.\cs{begin}|{envname}|と内容の間に\cs{label}\marg{label}とするのが良いでしょう.
%
%\subsection{askwoption}
%上で1つ紹介しましたが,\cs{askwoption}命令はパッケージオプションのように多数の引数を同時に指定可能です.例えば\cs{askwoption}|{thnnumber=3,thmlinebreak}|のようになります.ここでは\cs{askwoption}命令に指定できる引数を説明します.
%\subsubsection{thmlinebreak}
%各種定理環境の開始時において,自動で改行するように設定します.以下に例を示します.左がthmlinebreakが有効で,右が無効です.
%\makeatletter
%\begin{multicols}{2}
%	\@thmlinebreaktrue
%	\begin{proof}
%		吾輩は猫である。名前はまだ無い。どこで生れたかとんと見当がつかぬ。何でも薄暗いじめじめした所でニャーニャー泣いていた事だけは記憶している。
%	\end{proof}
%\@thmlinebreakfalse
%\begin{proof}
%	吾輩は猫である。名前はまだ無い。どこで生れたかとんと見当がつかぬ。何でも薄暗いじめじめした所でニャーニャー泣いていた事だけは記憶している。
%\end{proof}
%	\end{multicols}
%\subsubsection{zerostart}
%その名の通りカウンタを0から始めるようにします.ここではsection, figure, table, footnoteと\cs{makethm},\cs{newtheorem}が対象です.またカウンタの親子関係について,親カウンタがインクリメントされるとき子カウンタがリセットされる場合,子カウンタを0から開始されるように設定します.
%
%注意として,\cs{part}はデフォルトではローマ数字を使用するため0から開始するように設定してしまうと不具合が起こりますから,既定では設定から外しています.そのためもしpartカウンタの表示形式を変更したうえで,0から始めたいのであればプリアンブルで\cs{setcounter}|{part}{-1}|とするとよいでしょう(もちろん,ここで提供していないカウンタも同様の操作で0から始まるように設定することはこのオプションを使わずとも可能です).
%\subsubsection{sectionmark}
%節番号(\cs{section},\cs{subsection}など)における表示を変更し,例えば\S 1.3のように通し番号の前に\quo{\S}を追加します.
%
%\subsubsection{dottedtoc}
%このオプションは目次(\cs{tableofcontents})において,部(\cs{part})について見出しとページの間に下付きの点線を表示します.
%またsectionmarkオプションを有効にしているとき,節部分を調整します.
%
%\subsection{Some macros}
%\subsubsection{Macro to use in preamble}

%\macro{setnumdepth}enumerate環境はネスト(入れ子に)することが可能ですが,パッケージ等で拡張していない状態では5階層以降の深さにするとエラーを出します.そこで\cs{setnumdepth}\marg{num}とすると\meta{num}階層までのネストが可能になります(このマクロは試作品であまりデバッグができていません).
%
%\macro{addtoreset}単純に\LaTeX のマクロである\cs{@addtoreset}を\cs{makeatletter}下以外でも使えるようにしたものが\cs{addtoreset}です.\cs{addtoreset}\marg{counter1}\marg{counter2}のように使用し,\meta{counter1}は\meta{counter2}がインクリメントするごと(正確には\cs{stepcounter}によってインクリメントされたとき)にリセットされるようになります.
%\subsubsection{Macro for Document}
%\macro{part}本パッケージでは\cs{part}をjsarticle.clsに定義されているものをベースにすこし変更を加えています.単に\cs{part}\marg{part title}とするときには以前と同様の動作をしますが,省略可能引数について仕様の変更を与えています.従来の\cs{part}\oarg{toc title}\marg{part title}と同等の機能は\cs{part}\oarg{toc title}\texttt{[]}\marg{part title}とすることで得られます.単に\cs{part}\oarg{english title}\marg{part title}とすると次のようになります.
%
%\vspace*{1zh}
%\makeatletter
%{
%\def\prepartname{第}
%\def\postpartname{部}
%\def\thepart{\Roman{part}}
%\def\headfont{\bfseries}
%\parindent\z@
%\raggedright
%\interlinepenalty \@M
%\normalfont
%\setlength{\m@syu@length}{\textwidth}
%\settowidth{\m@syu@length@}{\huge\hspace*{.5em}\headfont\prepartname\thepart\postpartname}
%\addtolength{\m@syu@length}{-\m@syu@length@}
%\settowidth{\m@syu@length@}{\huge\headfont part title}
%\addtolength{\m@syu@length}{-\m@syu@length@}
%\addtolength{\m@syu@length}{-1em}
%\begin{tabular}{@{\vrule width 2pt}c}
%	\huge\hspace*{.5em}\headfont\prepartname\thepart\postpartname\hspace{\m@syu@length}\huge \headfont part title\\[10pt]
%	\setlength{\m@syu@length}{\textwidth}
%	\settowidth{\m@syu@length@}{\Large---\textsl{english title}}
%	\addtolength{\m@syu@length}{-\m@syu@length@}
%	\addtolength{\m@syu@length}{-1em}
%	\hspace{\m@syu@length}\Large---\textsl{english title}\\
%	\noalign{\hrule width \textwidth height 2pt}
%\end{tabular}
%}
%\vspace*{2zh}
%\makeatother
%
%この書式で\cs{part}を用いる場合\cs{newpage}を前置して予め改ページしておくことを強く推奨します.
%
%\macro{thepartchange}例えば\cs{thepart}を\cs{arabic}|{part}|などと再定義して,部番号をアラビア数字で表示しているとしましょう.これをアルファベットに変えようと思った場合\cs{renewcommand}|{\thepart}{\Alph{section}}|\cs{setcounter}|{part}{0}|とすればうまくいきますが,これではhyperlinkを使っているときに不具合が起こります.そこで\cs{thepartchange}命令を使うと\textbf{一度まで}不具合を回避しながら表示形式を変更できます.デフォルトで\cs{thepartchange}を用いると\cs{Alph}で表示する扱いになります.省略可能引数でroman,Roman,arabic,alph,kanziが指定可能で,それぞれ小文字のローマ数字,大文字のローマ数字,アラビア数字,小文字アルファベット,漢数字に対応します(もちろん同時に指定できるのは1つです).
%
%\macro{symlist}このマクロは\cs{symlist}\marg{symbol}\marg{description}のように使い,記号の説明に関する次のような書式を提供します.
%
%\symlist{symbol}{description}
%
%\macro{namelabel}\cs{namelabel}は\cs{namelabel}\marg{name}\marg{year of birth}\marg{year of death}のように使い,人名を脚注として出力しそのデータを内部に格納します.例えば;\\
%\cs{namelabel}|{Alexander Grothendieck}{1928}{2014}|\\
%のようにすることで,\namelabel{Alexander Grothendieck}{1928}{2014}となります.まだ亡くなっていない人物の場合は没年を空欄にして\cs{namelabel}\marg{year of death}\verb|{}|としてください.またitembox環境などの最中で脚注を使うと領域内にフッターが作成されますが,紙面下部のフッターに脚注を載せたい場合は,環境内の脚注をつけたい部分に\cs{footnotemark}を記述し環境を出た後に\cs{footnotetext}\marg{body}とすればうまくいく,という小技(?)がありますが,\cs{namelabel}で同様のことを行いたい場合は環境内に\cs{footnotemark}を記述して,環境を出た後に\cs{namelabel}$\ast$\marg{name}\marg{born}\marg{death}としてください.
%
%\macro{phantomnamelabel}\cs{phantomnamelabel}は脚注に出力せずにデータの格納のみを行います.
%
%\macro{namelabelOP}このマクロは今まで宣言した\cs{namalebal}によって格納された人名データを生年によって並び替え,出力します.次に宣言するダミーデータを並び替えてみましょう.
%
%\cs{phantomnamelabel}\verb|{dummy1}{1960}{2018}|\par
%\cs{phantomnamelabel}\verb|{dummy2}{1764}{1840}|\par
%\cs{phantomnamelabel}\verb|{dummy3}{1757}{1860}|\par
%\cs{phantomnamelabel}\verb|{dummy4}{2001}{}|
%
%
%\phantomnamelabel{dummy1}{1960}{2018}\phantomnamelabel{dummy2}{1764}{1840}\phantomnamelabel{dummy3}{1757}{1860}\phantomnamelabel{dummy4}{2001}{}
%\hrulefill
%
%\namelabelOP
%
%\hrulefill
%
%このようになります(先程例で使用したGrothendieckも並び替えられてしまっていますが).データ量が膨大になってきたときはmulticolパッケージを用いて;
%
%\cs{begin}|{multicols}{2}|
%
%\cs{namelabelOP}
%
%\cs{end}|{multicols}|
%
%などとするとよいでしょう.
%\subsubsection{Macros for text and formulas}
%
% 小技集です.相当昔のマクロも含まれるため(今まで以上に)行儀の悪いマクロ定義になっている可能性があります.
%
%\macro{quo}\cs{quo}\marg{arg}のように使い,\quo{arg}を出力します.
%
%\macro{uml}\cs{uml}命令は\cs{uml}{Alphabet}としてウムラウトを出力します.例えば\cs{uml}|{o}|で\uml{o}となります.
%
%
%\macro{mkset}\cs{mkset}命令は数式環境内で\cs{mkset}\marg{arg1}\marg{arg2}のようにして集合を記述します.例えば|$\mkset{a\in A}{f(a)=0}$|で$\mkset{a\in A}{f(a)=0}$と出力します.
%
%\macro{nitem}\macro{ntimes}\macro{nplace}\cs{nitem}命令は\cs{nitem}\oarg{alph}\marg{arg}のようにして繰り返しを記述します.\meta{alph}を省略すると,"n"であると解釈されます.すなわち|\nitem{\alpha}|では$\nitem{\alpha}$となり,|\nitem[k]{\alpha}|では$\nitem[k]{\alpha}$と出力します.また1から始めるのではなく任意の値から始めたい場合,例えば$\nitem<r>[r+n]{\alpha}$を出力するには\cs{nitem} |<r>[r+n]{\alpha}|のようにします.これの類似として次のコマンド\cs{ntimes}, \cs{nplace}が用意されています.\cs{ntimes}の書式は\cs{ntimes}\marg{num}\marg{arg}で,\meta{num}には2以上の整数を,\meta{arg}には繰り返したいものを記述します.このコマンドは区切りなしに\meta{num}回の\meta{arg}を出力します.例えば|\ntimes{5}{\alpha}|で$\ntimes{5}{\alpha}$となります.また\cs{nplace}命令は\cs{nitem}においてアルファベットでなく数字を指定するもので,区切り付きで出力します.書式は\cs{nplace}\marg{num}{\marg{arg}で\meta{num}は省略できません.
%
%\macro{nxcell}\macro{ses}\textbf{これらのマクロはTi\textit{k}Z--cdパッケージを前提にします.} 可換図式を書く際に記述を簡単にするコマンドをいくつか用意しています.\cs{nxcell}は\cs{nxcell}\oarg{label}のように使い, Ti\textit{k}Z--cdでの\cs{arrow}[r]\&と等価です.省略可能引数\oarg{label}を指定した場合には\cs{arrow}[r,"label"]\&として働きます.ただし次のセルを何も指定しなくてもエラーを出さないように|{}|を次のセルに配置します.例えば次の例を見て下さい.
%
%\cs{begin}|{tikzcd}|
%
%|0\nxcell A_1 \nxcell[f] A_2 \nxcell[g] A_3\nxcell 0\\|
%
%|0\nxcell A_1 \nxcell[f] A_2 \nxcell[g] A_3\nxcell |
%
%\cs{end}|{tikzcd}|
%
%\begin{tikzcd}
% 0\nxcell A_1 \nxcell[f] A_2 \nxcell[g] A_3\nxcell 0\\
% 0\nxcell A_1 \nxcell[f] A_2 \nxcell[g] A_3\nxcell 
%\end{tikzcd}
%
%また\cs{ses}は短完全列(Short exact sequence)の出力を支援します.具体的には\cs{ses}\oarg{1st label}\oarg{2nd label}\marg{object1}\marg{object2}\marg{object3}を書式とします.tikzcd環境内\textbf{ではなく}数式モード内で使用してください.次の例を見てください.
%
%短完全列|$\ses{A_1}{A_2}{A_3}$|において……
%
%-->
%
%短完全列$\ses{A_1}{A_2}{A_3}$において……
%
%\vspace*{1zh}
%
%|\[\ses[\varphi][\psi]{M_1}{M_2}{M_3}\]|
%
%-->
%
%\[\ses[\varphi][\psi]{M_1}{M_2}{M_3}\]
%
%
%\vspace*{1zh}
%
%短完全列|$\ses[f]{A_1}{A_2}{A_3}$|において……
%
%-->
%
%短完全列$\ses[f]{A_1}{A_2}{A_3}$において……
%
%
%\vspace*{1zh}
%
%短完全列|$\ses[][g]{A_1}{A_2}{A_3}$|において……
%
%-->
%
%短完全列$\ses[][g]{A_1}{A_2}{A_3}$において……
%\subsubsection{Macros for to write \TeX documents}
%
%
%\macro{cmd}本ドキュメントのように,文章中で\TeX のコントロールシークエンスなどを説明したい際に用いるコマンドです.\cs{cmd}\marg{tokenname}はタイプライタ体で\cmd{tokenname}を印字します.
%
%\macro{showme}あるコマンド\cs{controlsequence}の定義を知りたい場合に使用するコマンドです.使用例を以下に掲示します.
%
%\cs{showme}\verb|{expandafter}|\par
%-->
%
%\showme{expandafter}
%
%\cs{showme}\verb|{TeX}|\par
%-->
%
%\showme{TeX}
%
%\cs{showme}\verb*|{TeX }|\par
%-->
%
%\showme{TeX }
%
%\cs{showme}\verb*|{TeX{}}|\par
%%-->
%
%\showme{TeX{}}
%
%\cs{showme}|{TeXnichian}|\par
%-->
%
%\showme{TeXnichian}
%
%このように,コントロールシークエンス名に\textvisiblespace が含まれたトークンの定義を調べたい場合には\textvisiblespace の入るべき位置に|{}|を挿入してください.
%\subsubsection{Environment}
%本パッケージではいくつかの環境が新しく定義されています.それを紹介しましょう.
%
%まずは箇条書きを与える環境で,romanitemize, circitemize, numitemize, step環境です.使用方法はenumerate環境と同じく\cs{item}を用いて箇条書きにします.そちらの使用方法を参考にしてください.使用例は次のようになります.
% 
%  romanitemize環境;
% \begin{romanitemize}
% \item This is a meaningless sample text.
% \item This is a meaningless sample text.
% \item This is a meaningless sample text.
% \end{romanitemize}
%  circleitemize環境;
% \begin{circleitemize}
	% \item This is a meaningless sample text.
	% \item This is a meaningless sample text.
	% \item This is a meaningless sample text.
	% \end{circleitemize}
%  numitemize環境;
% \begin{numitemize}
	% \item This is a meaningless sample text.
	% \item This is a meaningless sample text.
	% \item This is a meaningless sample text.
	% \end{numitemize}
%  step環境;
% \begin{step}
	% \item This is a meaningless sample text.
	% \item This is a meaningless sample text.
	% \item This is a meaningless sample text.
	% \end{step}
%
%\vspace*{2zh}
%
%  また同値条件の証明を平易にするeqv環境が実装されています.使用方法はenumerateなどと同じく\cs{item}で十分条件と必要条件を区切ります.基本的にはproof環境などの中での使用を想定されています.
%
% \begin{eqv}
% \item このように\item なります.
% \end{eqv}
%eqv環境は省略可能引数\meta{num}を使うことで次のような書式;\def\shorttext{This is a meaningless sample text.}^^A
%
%\cs{begin}|{eqv}[3]|
%
%\cs{item} This is a meaningless sample text.
%
%\cs{item} This is a meaningless sample text.
%
%\cs{item} This is a meaningless sample text.
%
%\cs{end}|{eqv}|
%
%\begin{eqv}[3]
% \item\shorttext\item\shorttext\item\shorttext
%\end{eqv}
%が使用可能になります.それだけでなく,省略可能引数を[\meta{num}->\meta{num}]のようにすることで,次のようにすることもできます;
%
%|\begin{eqv}[3]|
%
%|\item[1->3]|This is a meaningless sample text.
%
%|\item[3->1]| This is a meaningless sample text.
%
%|\item[1->2]| This is a meaningless sample text.
%
%|\item[2->1]|\shorttext
%
%|\end{eqv}|
%\begin{eqv}[3]
%	\item[3->1]\shorttext\item[3->1]\shorttext\item[1->2]\shorttext\item[2->1]\shorttext
%\end{eqv}
%
%\macro{eqvlabelset}また\cs{eqvlabelset}は\cs{thepartchange}と同じ形式で\cs{begin}|{eqv}|の前に用いることで,ラベルの表示形式を変更します.指定可能なものはroman, Roman, arabic, alph, Alph, kanziで,デフォルトではarabicとなっています.
%
%  最後の環境はdefiterm環境で,これまたenumerateとおなじく\cs{item}で区切ります.この環境は必須引数を取ります.使用例は以下で;
%
% \cs{begin}|{defiterm}|\marg{ARG}
%
% \cs{item} This is a meaningless sample text.
% 
% \cs{item} This is a meaningless sample text.
%
% \cs{end}|{defiterm}|
%
%  により;
% \begin{defiterm}{ARG}
	% 	\item \shorttext\item\shorttext
	% \end{defiterm}
%  を出力します.
%
%
%\StopEventually{}
%\section{Definition of macros}
% 以下のコードにはいまだドキュメントにまとめていない命令が多数含まれていますが,それらは開発中ということでお願いします(使用したときに不具合などあればお知らせください).
%     \begin{macrocode}

\RequirePackage{xkeyval}				
\RequirePackage{amsmath,amssymb,amsthm}	%%\let\@xp\expandafter
\RequirePackage{ascmac}				
\RequirePackage{bxghost}				
\RequirePackage{iftex} 
\RequirePackage{etoolbox}

\def\m@syu@elt{\relax}
\def\m@syu@thmelt{\relax}
\def\m@syu@thmtwoelt{\relax}
\def\m@syu@thmoneelt{\relax}
\def\m@syu@zero@elt{\relax}

\def\m@syu@delete@define#1{\g@addto@macro\m@syu@elt{\let#1=\relax}}
%%
%%///Define error message
%%

\def\@m@syu@toosmall{\PackageError{askw3.sty}{The setenum argument must be 5 or more}\@ehd}
\def\@m@syu@samename{\PackageError{askw3.sty}{This person is already registered}\@ehd}
\def\@m@syu@eqvlabel{\PackageError{askw3.sty}{Use the specified argument}\@ehd}
\def\@m@syu@nopluatexerr{\PackageError{askw3.sty}{This command in only available in LuaTeX, p-TeX}\@ehd} 
\def\@m@syu@tateerr{\PackageError{askw3.sty}{tate option is only available for p-TeX}\@ehd} 
\def\@m@syu@duplicateerr{\PackageError{askw3.sty}{This command can be used only once}\@ehd} 

\def\m@syu@oldcommand#1{\PackageWarning{askw3.sty}{Use of \protect#1\space is not recommended.}}
\def\@m@syu@notnamed{\PackageWarning{askw3.sty}{Person date is not registerd.}}
\def\@m@syu@alreadytitlesetted{\PackageWarning{askw3.sty}{Title data are already setted, but I updated them.}}
\def\m@syu@fontselected#1{\PackageInfo{askw3.sty}{Font ``\protect#1\space " is loaded.}}
\def\m@syu@packageloaded#1{\PackageInfo{askw3.sty}{Package ``\protect#1\space " is loaded.}}
%%
%%///End of define error message
%%

%%
%%///Define Package Option
%%
\def\addoption#1{
	\@xp\newif\csname if@#1\endcsname
	\csname @#1false\endcsname
	\DeclareOption{#1}{\csname @#1true\endcsname}
}

\m@syu@delete@define{\addoption}

\long\def\optiondef#1{
	\csname if@#1\endcsname
	\@xp\@firstoftwo
	\else
	\@xp\@secondoftwo
	\fi
}
\m@syu@delete@define{\optiondef}

\addoption{links}

\def\m@syu@addoption#1{%boolean switch
	\@xp\newif\csname if@#1\endcsname
	\csname @#1false\endcsname
	\define@key{askwoption}{#1}[true]{
		\csname @#1##1\endcsname
	}
}

\m@syu@addoption{thmlinebreak}
\m@syu@addoption{barrenew}
\m@syu@addoption{subsectionmark}
\m@syu@addoption{zerostart}
\m@syu@addoption{dottedtoc}
\m@syu@addoption{sectionmark}

\newcommand{\askwoption}[1]{
	\setkeys{askwoption}{#1}
}

\@onlypreamble\askwoption

%Define thmnumber
\newif\if@thmnumtwo
\newif\if@thmnumthr
\@thmnumtwofalse
\@thmnumthrfalse

\define@key{askwoption}{thmnumber}[1]{
	\m@syu=#1
	\ifnum\m@syu=1
	\@thmnumtwofalse
	\@thmnumthrfalse
	\else
	\ifnum\m@syu=2
	\@thmnumtwotrue
	\@thmnumthrfalse
	\else
	\ifnum\m@syu=3
	\@thmnumthrtrue
	\@thmnumtwofalse
	\fi\fi\fi
}


%---Enable option
\ProcessOptions

%---Define Option behaived
\optiondef{links}{
	\@ifpackageloaded{hyperref}{
		\ifptex
			\RequirePackage{pxjahyper} %p-TeX以外では読み込まない
			\m@syu@packageloaded{pxjahyper}
		\fi
	}{	
		\RequirePackage[hidelinks,hyperfigures]{hyperref}
		\m@syu@packageloaded{hyperref}
		\ifptex
			\RequirePackage{pxjahyper}
			\m@syu@packageloaded{pxjahyper}
		\fi
	}
}{}

\@ifpackageloaded{hyperref}{
	\def\m@syu@href{\refstepcounter{Item}\global\protected@edef\@currentHref{Item.\arabic{Item}}}
}{\def\m@syu@href{\relax}}

\AtEndPreamble{
	%%%thmnumber
	\if@thmnumthr
	\def\theequation{\thepart.\thesection.\arabic{equation}}
	\m@syu@thmelt
	\else
	\if@thmnumtwo
	\def\theequation{\thesection.\arabic{equation}}
	\m@syu@thmtwoelt
	\else
	\def\theequation{\arabic{equation}}
	\m@syu@thmoneelt
	\fi\fi
	%%%zerostart
	\if@zerostart
	\c@figure=\m@ne
	\c@table=\m@ne
	\c@footnote=\m@ne
	\c@section=\m@ne
	\@ifundefined{c@chapter}{}{\c@chapter=\m@ne}
	\@ifundefined{c@part}{}{\c@part=\m@ne}
	\def\@stpelt#1{\global \csname c@#1\endcsname -2\stepcounter {#1}}
	\m@syu@zero@elt
	\fi
	%%%sectionmark
	\if@sectionmark
	\def\@sect#1#2#3#4#5#6[#7]#8{%
		\if@subsectionmark
		\@xp\let\@xp\m@syu@tempa\csname the#1\endcsname
		\@xp\def\csname the#1\endcsname{\S~\m@syu@tempa}%
		\else
		\let\m@syu@tempa\thesection
		\def\thesection{\S~\m@syu@tempa}%
		\fi
		\ifnum #2>\c@secnumdepth
		\let\@svsec\@empty
		\else
		\refstepcounter{#1}%
		\protected@edef\@svsec{\@seccntformat{#1}\relax}%
		\fi
		\@tempskipa #5\relax
		\ifdim \@tempskipa<\z@
		\def\@svsechd{%
			#6{\hskip #3\relax
				\@svsec #8}%
			\csname #1mark\endcsname{#7}%
			\addcontentsline{toc}{#1}{%
				\ifnum #2>\c@secnumdepth \else
				\protect\numberline{\csname the#1\endcsname}%
				\fi
				#7}}%目次にフルネームを載せるなら #8
		\else
		\begingroup
		\interlinepenalty\@M
		#6{%
			\@hangfrom{\hskip #3\relax\@svsec}%
			#8\@@par}%
		\endgroup
		\csname #1mark\endcsname{#7}%
		\addcontentsline{toc}{#1}{%
			\ifnum #2>\c@secnumdepth \else
			\protect\numberline{\csname the#1\endcsname}%
			\fi
			#7}% 目次にフルネームを載せるならここは #8
		\fi
		\if@subsectionmark
		\@xp\let\csname the#1\endcsname\m@syu@tempa
		\let\m@syu@tempa\relax
		\else
		\let\thesection\m@syu@tempa
		\let\m@syu@tempa\relax
		\fi
		\@xsect{#5}%
	}
	\fi
	%%%dottedtoc
	\if@dottedtoc
	\def\l@part#1#2{%
		\ifnum \c@tocdepth >-2
		\addpenalty{\@secpenalty}%
		\addvspace{2.25em \@plus\p@}%
		\begingroup
		\parindent\z@
		\rightskip\@tocrmarg
		\parfillskip-\rightskip
		\leavevmode\headfont
		\setlength\@lnumwidth{4\zw}%
		\advance\leftskip\@lnumwidth 
		\hskip-\leftskip #1\nobreak
		\leaders\hbox{\normalfont$\m@th \mkern
			\@dotsep mu\hbox{.}\mkern \@dotsep mu$}\hfill\nobreak
		\hbox to\@pnumwidth{\hss#2} %%page を表示しないならコメントアウト
		\par
		\endgroup
		\fi
	}
	\if@sectionmark
	\def\l@section#1#2{%
		\ifnum \c@tocdepth >\z@
		\addpenalty{\@secpenalty}%
		\addvspace{1.0em \@plus\jsc@mpt}%
		\begingroup
		\parindent1.5em
		\rightskip\@tocrmarg
		\parfillskip-\rightskip
		\leavevmode\headfont
		\setlength\@lnumwidth{\jsc@tocl@width}\advance\@lnumwidth 2ex
		\advance\leftskip\@lnumwidth \hskip-\leftskip
		#1\nobreak\hfil\nobreak\parindent1.5em
		\hbox to\@pnumwidth{\hss#2}\par
		\endgroup
		\fi
	}
	\fi
	\fi
}
%%///End of Define Package option
%%
%---Make new counter,length
\newcount\m@syu
\newcount\m@syu@
\newcount\m@syu@@

\newcount\m@syu@name
\newcount\m@syu@sort@length
\newcount\c@m@syu@eqv
\newlength\m@syu@length
\newlength\m@syu@length@
\newlength\m@syu@length@@
\newlength\masyulengtha
\newlength\masyulengthb
\newlength\masyulengthc

\m@syu@name=\z@
\c@m@syu@eqv=\z@

%%%//Define theorem environment
\newcommand{\makethm}{\@ifstar{\makethm@star}{\makethm@nonstar}}
\@onlypreamble{\makethm}

\newcommand{\thmnotefontchange}[1]{\gdef\m@syu@thm@notefont{#1}}


\def\m@syu@punct{\relax}

\def\makethm@star#1#2{%
	\newenvironment{#1}[1][]{%
		\begin{itembox}[l]{#2\m@syu@thmlabel{##1}}
		}{\end{itembox}}%
}

\def\makethm@nonstar#1{%
	\let\@tempa\relax
	\def\@tempa{\@oparg{\makethm@{#1}}[]}%
	\@tempa
}

\def\makethm@#1[#2]#3{%
	\ifx\relax#2\relax
	\@ifundefined{c@#1}{%
		\newcounter{#1}%
		\g@addto@macro\m@syu@thmelt{%
			\@xp\def\csname the#1\endcsname{\thepart.\arabic{section}.\arabic{#1}}%
			\@addtoreset{#1}{section}}%
		\g@addto@macro\m@syu@thmtwoelt{%
			\@xp\def\csname the#1\endcsname{\arabic{section}.\arabic{#1}}%
			\@addtoreset{#1}{section}}%
		\g@addto@macro\m@syu@thmoneelt{%
			\@xp\def\csname the#1\endcsname{\arabic{#1}}%
			\@addtoreset{#1}{section}}%
		\g@addto@macro\m@syu@zero@elt{\setcounter{#1}{-1}}%
		\newenvironment{#1}[1][]{%
			\addtocounter{#1}{1}%
			\protected@edef\@currentlabel{#3\csname the#1\endcsname}%
			\begin{itembox}[l]{%
				#3\m@syu@punct\textit{\csname the#1\endcsname}\m@syu@thmlabel{##1}\m@syu@href}%
			}{\end{itembox}}%
	}{%
		\PackageError{askw.sty}{'#1' environment is already defined}\@eha
	}%
	\else
	\@ifundefined{c@#2}{\@nocounterr{#2}%
	}{%
		\newenvironment{#1}[1][]{%
			\addtocounter{#2}{1}%
			\protected@edef\@currentlabel{#3\csname the#2\endcsname}%
			\begin{itembox}[l]{%
				#3\textit{\csname the#2\endcsname}\m@syu@thmlabel{##1}\m@syu@href}%
			}{\end{itembox}}%
	}%
	\fi
}

\def\m@syu@thm@notefont{\fontseries\mddefault\upshape}

\def\thmhead@plain#1#2#3{%
	\m@syu@href
	\thmname{#1}\thmnumber{\@ifnotempty{#1}{ }\@upn{#2}}%
	\thm@notefont{\m@syu@thm@notefont}%
	\thmnote{ {\the\thm@notefont(#3)}}%
}

\def\@xthm#1#2[#3]{%
	\ifx\relax#3\relax
	\newcounter{#1}%
	\else
	\newcounter{#1}[#3]%
	\@xp\xdef\csname the#1\endcsname{\@xp\@nx\csname the#3\endcsname
		\@thmcountersep\@thmcounter{#1}}%
	\fi
	\g@addto@macro\m@syu@thmelt{%
		\@xp\def\csname the#1\endcsname{\thepart.\thesection.\arabic{#1}}%
		\@addtoreset{#1}{section}}%
	\g@addto@macro\m@syu@thmtwoelt{%
		\@xp\def\csname the#1\endcsname{\thesection.\arabic{#1}}%
		\@addtoreset{#1}{section}}%
	\g@addto@macro\m@syu@thmoneelt{%
		\@xp\def\csname the#1\endcsname{\arabic{#1}}%
		\@addtoreset{#1}{section}}%
	\g@addto@macro\m@syu@zero@elt{\setcounter{#1}{-1}}%
	\toks@{#2}%
	\@xp\xdef\csname#1\endcsname{%
		\@nx\@thm{%
			\let\@nx\thm@swap
			\if S\thm@swap\@nx\@firstoftwo\else\@nx\@gobble\fi
			\@xp\@nx\csname th@\the\thm@style\endcsname}%
		{#1}{\the\toks@}}%
}

\def\m@syu@thmlabel#1{%
	\def\m@syu@thm@{#1}%
	\ifx\m@syu@thm@\empty
	\relax
	\else
	%\nobreakspace(#1)
	(#1)%Zenkaku
	\fi
}

\newcommand{\thmnumonly}[1]{%
	\g@addto@macro\m@syu@thmelt{%
		\@xp\def\csname the#1\endcsname{\arabic{#1}}%
	}%
	\g@addto@macro\m@syu@thmtwoelt{%
		\@xp\def\csname the#1\endcsname{\arabic{#1}}%
	}%
	\g@addto@macro\m@syu@thmoneelt{%
		\@xp\def\csname the#1\endcsname{\arabic{#1}}%
	}%
}

\@onlypreamble{\thmnumonly}

\AtEndPreamble{%
	\def\@begintheorem#1#2[#3]{%
		\deferred@thm@head{%
			\the\thm@headfont\thm@indent
			\@ifempty{#1}{\let\thmname\@gobble}{\let\thmname\@iden}%
			\@ifempty{#2}{\let\thmnumber\@gobble}{\let\thmnumber\@iden}%
			\@ifempty{#3}{\let\thmnote\@gobble}{\let\thmnote\@iden}%
			\thm@swap\swappedhead
			\thmhead{#1}{#2}{#3}%
			\the\thm@headpunct\thmheadnl\hskip\thm@headsep}%
		\global\protected@edef\@currentlabel{#1#2}%
		\if@thmlinebreak\quad\par\fi}%
}

\renewenvironment{proof}[1][\proofname]{%
	\pushQED{\qed}%
	\normalfont \topsep6\p@\@plus6\p@\relax
	\trivlist
	%\interlinepenalty\@M
	\@itempenalty\@M 
	\item[\hskip\labelsep
	\itshape
	#1\@addpunct{\m@syu@afterpunct}]\quad
	\if@thmlinebreak
	\@xp\@firstoftwo
	\else
	\@xp\@secondoftwo
	\fi
	{\@ifnextchar\begin{\item}{\setlength{\itemindent}{1em}\item}}{}%
	}{\popQED\endtrivlist\@endpefalse}
	
	\newif\if@m@syu@eng
	\@m@syu@engfalse
	
	\def\m@syu@afterpunct{%
		\if@m@syu@eng
		.
		\else
		\textbf{.}
		\fi
	}
	
	\newenvironment{answer}[1][\textbf{解答}]{%
		\let\m@syu@qed\qedsymbol
		\def\qedsymbol{(解答終)}%
		\def\proofname@{#1}%
		\pushQED{\qed}%
		\normalfont \topsep6\p@\@plus6\p@\relax
		\trivlist
		\interlinepenalty\@M
		\@itempenalty\@M 
		\item[\hskip\labelsep
		\itshape
		#1\@addpunct{.}]\quad
		\if@thmlinebreak
		\@xp\@firstoftwo
		\else
		\@xp\@secondoftwo
		\fi
		{\@ifnextchar\begin{\item}{\setlength{\itemindent}{1em}\item}}{}%
		}{%
			\popQED\endtrivlist\@endpefalse
			\let\qedsymbol\m@syu@qed
		}
		
		%%
		%%///End of define theorem environment
		%%
		%---Rewrite \part 
		%%% \partの書き換えにより,本来の\part[<text>]{<text>}は\part[<text>][]{<text>}で得られるようになる.
		
		\def\part{%
			\if@noskipsec \leavevmode \fi
			\par
			\addvspace{4ex}%
			\if@english \@afterindentfalse \else \@afterindenttrue \fi
			\secdef\m@syu@part\m@syu@spart %\part{X}-> \m@syu@part[X]{X}
		}
		
		\def\m@syu@part{%
			\def\m@syu@finalrun{\m@syu@part@}%
			\@ifnextchar[{\m@syu@get@one}{%
				\def\m@syu@label@one{\empty}%
				\def\m@syu@label@two{\empty}%
				\m@syu@finalrun}%
		}
		
		\def\m@syu@part@{%
			\@xp\ifx\m@syu@label@two\empty
			\def\m@syu@part@eng{\m@syu@label@one}%
			\else
			\def\m@syu@part@eng{\m@syu@label@two}%
			\fi
			\@part[\m@syu@part@eng]%
		}
		
		\def\@part[#1]#2{%
			\@xp\ifx\m@syu@label@two\empty %if \part{text} or \part[text]{text}
			\def\m@syu@parttoc{#2}%
			\else
			\def\m@syu@parttoc{\m@syu@label@one}%
			\fi   %%%%%%%%%%%%%%%%%%%%%%%%%%%%%%%%%%%%%%%%%%%%%%%%%%%%%%%%%%%%%%%%%
			\ifx\m@syu@label@two\empty %if \part[text][]{text}
			\def\m@syu@part@chka{\relax}%
			\def\m@syu@part@chkb{\relax}%
			\else                      %if \part[text]{text} ,\part[X][Y]{Z}
			\def\m@syu@part@chka{#1}%
			\def\m@syu@part@chkb{#2}%
			\fi
			\def\m@syu@part@tempa{#2}%
			\ifx\m@syu@label@one\m@syu@part@tempa %if \part{text}
			\@xp\ifx\m@syu@label@two\empty
			\def\m@syu@part@chka{\relax}%
			\def\m@syu@part@chkb{\relax}%
			\else\fi
			\else\fi
			\ifx\m@syu@part@chka\m@syu@part@chkb
			\else\newpage\thispagestyle{plain}%
			\fi
			\ifnum \c@secnumdepth>\m@ne
			\refstepcounter{part}%
			\ifx\m@syu@partchanged\relax
			\else\refstepcounter{m@syu@part}%
			\fi
			\addcontentsline{toc}{part}{%
				\prepartname\thepart\postpartname\hspace{1em}\m@syu@parttoc}{}%
			\else
			\addcontentsline{toc}{part}{#2}{}%
			\fi
			\markboth{\prepartname\thepart\postpartname\hspace{1em}\m@syu@parttoc}{}%
			{\parindent\z@
				\raggedright
				\interlinepenalty \@M
				\normalfont
				\ifnum \c@secnumdepth >\m@ne
				\ifx\m@syu@part@chka\m@syu@part@chkb
				\Large\headfont\prepartname\thepart\postpartname
				\par\nobreak
				\huge \headfont #2
				\else 
				\setlength{\m@syu@length}{\textwidth}%
				\settowidth{\m@syu@length@}{\huge\hspace*{.5em}\headfont\prepartname\thepart\postpartname}%
				\addtolength{\m@syu@length}{-\m@syu@length@}%
				\settowidth{\m@syu@length@}{\huge\headfont #2}%
				\addtolength{\m@syu@length}{-\m@syu@length@}% 
				\addtolength{\m@syu@length}{-1em}% part左の柱のぶん
				\begin{tabular}{@{\vrule width 2pt}c}%
					\huge\hspace*{.5em}\headfont\prepartname\thepart\postpartname\hspace{\m@syu@length}\huge \headfont #2\\[10pt]%
					\setlength{\m@syu@length}{\textwidth}%
					\settowidth{\m@syu@length@}{\Large---\textsl{#1}}%
					\addtolength{\m@syu@length}{-\m@syu@length@}%
					\addtolength{\m@syu@length}{-1em}%
					\hspace{\m@syu@length}\Large---\textsl{#1}\\
					\noalign{\hrule width \textwidth height 2pt}%
				\end{tabular}
				\fi
				\fi
				\markboth{\prepartname\thepart\postpartname\hspace{1em}#2}{}\par}%
			\nobreak
			\vskip 3ex
			\@afterheading
		}
		
		\def\m@syu@spart#1{%
			{%
				\parindent \z@ \raggedright
				\interlinepenalty \@M
				\normalfont
				\huge \headfont #1\par}%
			\markboth{#1}{#1}%
			\nobreak
			\vskip 3ex
			\@afterheading
		}
		%---Internal definition (only use in this package)
		
		\def\equiv@label{%
			\m@syu=\@ne\relax
			\def\item{%
				\ifnum\m@syu@@=\@enumdepth
				\ifnum \m@syu>\@ne\relax
				\par\noindent
				\fi
				\bgroup\interlinepenalty\@M			
				\ifnum \m@syu=\@ne\relax
				\mbox{($\Longrightarrow$)}%
				\else
				\mbox{($\Longleftarrow$)}%
				\fi
				\global\advance\m@syu\@ne\relax\\\quad\egroup
				\else
				\m@syu@eqv@item
				\fi
			}%
		}
		
		\def\equiv@label@roman{\romannumeral}
		\def\equiv@label@Roman{\@xp\@Roman}
		\def\equiv@label@arabic{\relax}
		\def\equiv@label@alph{\@xp\@alph}
		\def\equiv@label@Alph{\@xp\@Alph}
		\def\equiv@label@kanzi{\kansuji}
		
		\def\equiv@temp{\romannumeral}
		
		\def\m@syu@eqv@parametorcheck#1{\@m@syu@eqv@parametorcheck#1->->\@nil}
		\def\@m@syu@eqv@parametorcheck#1->#2->#3\@nil{
			\def\m@syu@eqv@firstparam{#1}
			\def\m@syu@eqv@secondparam{#2}
		}
		
		\def\equiv@label@{%
			\m@syu=\@ne\relax
			\renewcommand{\item}[1][0]{%
				\ifnum \@enumdepth=\m@syu@@\relax
				\ifnum \m@syu>\@ne\relax
				\par\noindent
				\fi
				\m@syu@eqv@parametorcheck{##1}\relax
				\ifx\m@syu@eqv@secondparam\empty
				\ifnum \m@syu@eqv@firstparam=\z@\relax
				\m@syu@=\m@syu\relax
				\global\advance\m@syu@\@ne\relax
				\else
				\m@syu=\m@syu@eqv@firstparam\relax
				\m@syu@=\m@syu\relax
				\global\advance\m@syu@\@ne\relax
				\fi
				\else
				\m@syu=\m@syu@eqv@firstparam\relax
				\m@syu@=\m@syu@eqv@secondparam\relax
				\fi
				\bgroup\interlinepenalty\@M
				\ifnum \m@syu=\c@m@syu@eqv\relax
				\mbox{(\equiv@temp\the\m@syu)~$\Longrightarrow$~(\equiv@temp\the\@ne)}%
				\else
				\mbox{(\equiv@temp\the\m@syu)~$\Longrightarrow$~(\equiv@temp\the\m@syu@)}%
				\fi
				\global\advance\m@syu\@ne\relax\\\quad\egroup
				\else
				\m@syu@eqv@item
				\fi
			}%
		}
		
		\def\namelabel@push#1#2#3{%
			\ifnum\m@syu@name=\z@
			\def\m@syu@named{\relax}%
			\global\advance\m@syu@name\@ne
			\@xp\def\csname m@syu@name@\the\m@syu@name\endcsname{#1}%
			\@xp\def\csname m@syu@born@\the\m@syu@name\endcsname{#2}%
			\@xp\def\csname m@syu@died@\the\m@syu@name\endcsname{#3}%
			\else
			\def\m@syu@namelabelchk{#1}%
			\global\advance\m@syu@name\@ne
			\m@syu=\@ne
			\@whilenum\m@syu<\m@syu@name
			\do{%
				\@xp\ifx\csname m@syu@name@\the\m@syu\endcsname\m@syu@namelabelchk
				\@m@syu@samename
				\fi
				\global\advance\m@syu\@ne
			}%
			\@xp\def\csname m@syu@name@\the\m@syu@name\endcsname{#1}%
			\@xp\def\csname m@syu@born@\the\m@syu@name\endcsname{#2}%
			\@xp\def\csname m@syu@died@\the\m@syu@name\endcsname{#3}%
			\fi				
		}
		
		\def\namelabel@#1#2#3{%
			\namelabel@push{#1}{#2}{#3}%
			\footnotetext{#1,#2-#3}%
		}
		
		\def\namelabel@@#1#2#3{%
			\namelabel@push{#1}{#2}{#3}%
			\footnote{#1,#2-#3}%
		}
		
		\def\m@syu@finalrun{\relax}
		
		\def\m@syu@get@one[#1]{%
			\def\m@syu@label@one{#1}%
			\@ifnextchar[{\m@syu@get@two}{%
				\def\m@syu@label@two{\empty}%
				\m@syu@finalrun
			}%
		}
		
		\def\m@syu@get@two[#1]{%
			\def\m@syu@label@two{#1}%
			\m@syu@finalrun
		}
		
		\AtEndPreamble{\if@barrenew\let\bar\overline\fi}
		
		%%
		%%///Define command which used preamble
		\newcommand{\setenumdepth}[1]{%
			\ifnum #1<5 \@m@syu@toosmall
			\else
			\m@syu=#1\relax
			\def\list##1##2{%
				\ifnum \@listdepth >\m@syu
				\@toodeep
				\else
				\global\advance\@listdepth\@ne
				\fi
				\rightmargin\z@
				\listparindent\z@
				\itemindent\z@
				\csname @list\romannumeral\the\@listdepth\endcsname
				\def\@itemlabel{##1}%
				\let\makelabel\@mklab
				\@nmbrlistfalse
				##2\relax
				\@trivlist
				\parskip\parsep
				\parindent\listparindent
				\advance\linewidth -\rightmargin
				\advance\linewidth -\leftmargin 
				\advance\@totalleftmargin \leftmargin
				\parshape \@ne \@totalleftmargin \linewidth
				\ignorespaces
			}
			\m@syu=\thr@@\relax
			\@whilenum \m@syu<#1 \relax                     
			\do{\@definecounter{enum\romannumeral\the\m@syu}%
				\advance\m@syu\@ne}%
			\@definecounter{enum\romannumeral\the\m@syu}%
			\def\enumerate{%
				\ifnum \@enumdepth >#1 \@toodeep\else
				\advance\@enumdepth \@ne
				\edef\@enumctr{enum\romannumeral\the\@enumdepth}\fi
				\@ifnextchar[{\@@enum@}{\@enum@}}%
			\fi
		}
		
		\newcommand{\myheader}[1]{%
			\pagestyle{fancy}%
			\def\sectionmark##1{\markright{%
					\ifnum \c@secnumdepth >\z@ \thesection \hskip1\zw\fi
					##1}}%
			\lhead{\nouppercase{\leftmark}}%
			\chead{#1}%
			\rhead{\nouppercase{\rightmark}}%
			\fancyfoot[C]{\thepage}%
		}
		
		\newif\ifm@syu@setmytitle
		\m@syu@setmytitlefalse
		
		\define@key[m@syu]{setmytitle}{author}[Jone Doe]{\def\m@syu@author{#1}}
		\define@key[m@syu]{setmytitle}{date}[\today]{\def\m@syu@date{#1}}
		\define@key[m@syu]{setmytitle}{none}[\relax]{
			\ifx#1\relax
			\else
			\DeclareRobustCommand{\m@syu@date}{}
			\DeclareRobustCommand{\m@syu@author}{}
			\fi
		}
		
		\presetkeys[m@syu]{setmytitle}{author,date}{}
		
		\newcommand{\setmytitle}[1]{%
			\ifm@syu@setmytitle
			\@m@syu@alreadytitlesetted
			\fi
			\setkeys[m@syu]{setmytitle}{#1}%
			\m@syu@setmytitletrue	
		}
		
		\newcommand{\mytitle}{\@ifnextchar[{\setmytitle@sec}{\m@syu@mytitle}}
		\def\setmytitle@sec[#1]{%
			\ifm@syu@setmytitle
			\@m@syu@alreadytitlesetted
			\fi
			\setkeys[m@syu]{setmytitle}{#1}%
			\m@syu@setmytitletrue	
			\m@syu@mytitle
		}
		
		\AtEndPreamble{%
			\@ifpackageloaded{fancyhdr}{%
				\def\m@syu@mytitle#1{%
					\ifm@syu@setmytitle
					\else\setkeys[m@syu]{setmytitle}{}%
					\fi
					\title{#1}\author{\m@syu@author}\date{\m@syu@date}%
					\maketitle
					\myheader{#1}%
					\thispagestyle{empty}%
					\c@page=\z@
				}%
			}{%
				\def\m@syu@mytitle#1{%
					\ifm@syu@setmytitle
					\else\setkeys[m@syu]{setmytitle}{}%
					\fi
					\title{#1}\author{\m@syu@author}\date{\m@syu@date}%
					\maketitle
					\c@page=\z@
				}%
			}%
		}%
		
		
		%%
		%%///End of defining about command which used preamble
		%%
		%---About consecutive number
		\let\m@syu@partchanged\relax
		\newcounter{m@syu@part}
		\newcommand{\thepartchange}[1][Alph]{%
			\setcounter{m@syu@part}{0}%
			\def\m@syu@partchanged{changed}%
			\let\m@syu@orig@thepart=\thepart
			\newcount\m@syu@part@save
			\m@syu@part@save=\c@part
			\@xp\let\@xp\partchange@temp\csname equiv@label@#1\endcsname\relax
			\ifx\partchange@temp\equiv@label@roman
			\else
			\ifx\partchange@temp\equiv@label@Roman
			\else
			\ifx\partchange@temp\equiv@label@kanzi
			\else
			\ifx\partchange@temp\equiv@label@arabic
			\else
			\ifx\partchange@temp\equiv@label@Alph
			\else
			\ifx\partchange@temp\equiv@label@alph
			\else
			\m@syu@eqvlabel %Error message
			\fi\fi\fi\fi\fi\fi
			\gdef\thepart{\partchange@temp\the\c@m@syu@part}%
		}
		\newcommand{\thepartchangefinish}{%
			\let\m@syu@partchanged\relax
			\let\thepart\m@syu@orig@thepart
			\c@part=\m@syu@part@save
		}
		%%///Define command which used article
		%---To simplify input
		%%%%General math command
		\newcommand{\mkset}[2]{\left\{#1\mathrel{}\middle|\mathrel{}#2\right\}}
		
		\newcommand{\nitem}{\@ifnextchar<{\@nitem@}{\def\nitem@temp{1}\@nitem}}
		
		\def\@nitem@<#1>{%
			\def\nitem@temp{#1}%
			\@nitem}
		
		\newcommand{\@nitem}[2][n]{#2_\nitem@temp,\dots,#2_{#1}}
		
		\newcommand{\nplace}[2]{%
			\ifnum #1>\z@ \relax
			\m@syu=\@ne\relax
			\@whilenum\m@syu<#1\relax
			\do{{#2}_{\the\m@syu},\advance\m@syu\@ne\relax}%
			{#2}_{#1}%
			\fi
		}
		
		\newcommand{\ntimes}[2]{%
			\m@syu=\@ne\relax
			\@whilenum\m@syu<#1\relax
			\do{#2\relax\advance\m@syu\@ne}%
			#2
		}
		
		\newcommand{\spanned}{\@ifnextchar[{\m@syu@spanned@}{\m@syu@spanned@@}}
		\newcommand{\m@syu@spanned@}[2][]{\left\langle #2\mathrel{}\middle|\mathrel{}#1\right\rangle}
		\newcommand{\m@syu@spanned@@}[1]{\left\langle #1\right\rangle}
		
		\newcommand{\ithone}[1][i]{\mathop{\vphantom=\hat{1}}\limits^{\scriptscriptstyle{#1}}}
		
		\newcommand{\ses}{%
			\def\m@syu@finalrun{\ses@making}%
			\@ifnextchar[{\m@syu@get@one}{%
				\def\m@syu@label@one{\empty}%
				\def\m@syu@label@two{\empty}%
				\m@syu@finalrun}%
		}
		
		\def\ses@making#1#2#3{%
			\begin{tikzcd}%
				0
				\arrow[r]\pgfmatrixnextcell#1\arrow[r,"\m@syu@label@one"]
				\pgfmatrixnextcell#2\arrow[r,"\m@syu@label@two"]
				\pgfmatrixnextcell#3\arrow[r]\pgfmatrixnextcell
				0
			\end{tikzcd}%
		}
		
		\newcommand{\nxcell}{\@ifnextchar[{\nxcell@label}{\nxcell@nolabel}}
		
		\def\nxcell@label[#1]{{}\arrow[r,"#1"]\pgfmatrixnextcell{}}
		\def\nxcell@nolabel{{}\arrow[r]\pgfmatrixnextcell{}}
		
		\newcommand{\darrow}{\@ifnextchar[{\darrow@label}{\darrow@nolabel}}
		
		\def\darrow@label[#1]{{}\arrow[d,"#1"]}
		\def\darrow@nolabel{{}\arrow[d]}
		%---Addition function 
		
		\newcommand{\quo}[1]{``#1''}
		\newcommand{\uml}[1]{\"#1}
		
		\newcommand{\addtoreset}[2]{\@addtoreset{#1}{#2}}
		
		\newcommand{\romannum}[1]{%
			\m@syu=#1\relax
			\textrm{(\romannumeral\the\m@syu)}}
		
		\newcommand{\mif}[4]{%Zenkaku
			\begin{cases}
				#1&(#2\text{のとき.})\\
				#3&(#4\text{のとき.})
			\end{cases}%
		}
		
		\newcommand{\reitem}[1]{%
			\@xp\def\csname labelenum\romannumeral\the\@enumdepth\endcsname
			{\@xp\m@syu\@xp=\csname c@enum\romannumeral\the\@enumdepth\endcsname\relax
				(#1{\the\m@syu})%
			}%
		}
		
		\newcommand{\symlist}[2]{%
			#1
			\setlength{\m@syu@length}{3cm}%
			\settowidth{\m@syu@length@}{\mbox{#1}}%
			\addtolength{\m@syu@length}{-\m@syu@length@}%
			\leaders\hbox{\normalfont$\m@th \mkern%
				\@dotsep mu\hbox{.}\mkern \@dotsep mu$}\hskip\m@syu@length
			#2\par
		}
		
		\newcommand{\namelabel}{\@ifstar{\namelabel@}{\namelabel@@}}
		\newcommand{\phantomnamelabel}[3]{\namelabel@push{#1}{#2}{#3}}
		\newcommand{\hnamelabel}[4][\m@syu@name@nooption]{%
			\def\m@syu@name@nooption{#2}%
			#2\ (#3-#4)\namelabel@push{#1}{#3}{#4}%
		}
		
		\newcommand{\namelabelOP}{%
			\@ifundefined{m@syu@named}{\@m@syu@notnamed}{%
				\newcount\c@m@syu@borna
				\newcount\c@m@syu@bornb
				\newcount\c@m@syu@dieda
				\newcount\c@m@syu@diedb
				\@tempcnta=\@ne
				\m@syu@sort@length=\m@syu@name
				\@whilenum\@tempcnta<\m@syu@name\do{%
					\m@syu=\@ne
					\m@syu@=\@ne
					\@whilenum\m@syu<\m@syu@sort@length
					\do{% 
						\global\advance\m@syu@\@ne
						\@xp\c@m@syu@borna\@xp=\csname m@syu@born@\the\m@syu\endcsname\relax
						\@xp\c@m@syu@bornb\@xp=\csname m@syu@born@\the\m@syu@\endcsname\relax
						\ifnum\c@m@syu@borna=\c@m@syu@bornb
						\xdef\m@syu@emptychka{\csname m@syu@died@\the\m@syu\endcsname}
						\xdef\m@syu@emptychkb{\csname m@syu@died@\the\m@syu@\endcsname}
						\ifx\m@syu@emptychka\empty
						\c@m@syu@dieda=\@M
						\ifx\m@syu@emptychkb\empty
						\c@m@syu@diedb=\@M
						\else
						\@xp\c@m@syu@diedb\@xp=\csname m@syu@died@\the\m@syu@\endcsname\relax
						\fi
						\else
						\@xp\c@m@syu@dieda\@xp=\csname m@syu@died@\the\m@syu\endcsname\relax
						\ifx\m@syu@emptychkb\empty
						\c@m@syu@diedb=\@M
						\else
						\@xp\c@m@syu@diedb\@xp=\csname m@syu@died@\the\m@syu@\endcsname\relax
						\fi
						\fi
						\ifnum\c@m@syu@dieda>\c@m@syu@diedb
						\xdef\m@syu@nametemp{\csname m@syu@name@\the\m@syu\endcsname}
						\xdef\m@syu@borntemp{\csname m@syu@born@\the\m@syu\endcsname}
						\xdef\m@syu@diedtemp{\csname m@syu@died@\the\m@syu\endcsname}
						\@xp\xdef\csname m@syu@name@\the\m@syu\endcsname{\csname m@syu@name@\the\m@syu@\endcsname}
						\@xp\xdef\csname m@syu@born@\the\m@syu\endcsname{\csname m@syu@born@\the\m@syu@\endcsname}
						\@xp\xdef\csname m@syu@died@\the\m@syu\endcsname{\csname m@syu@died@\the\m@syu@\endcsname}
						\@xp\xdef\csname m@syu@name@\the\m@syu@\endcsname{\m@syu@nametemp}
						\@xp\xdef\csname m@syu@born@\the\m@syu@\endcsname{\m@syu@borntemp}
						\@xp\xdef\csname m@syu@died@\the\m@syu@\endcsname{\m@syu@diedtemp}
						\fi
						\else
						\ifnum\c@m@syu@borna>\c@m@syu@bornb
						\xdef\m@syu@nametemp{\csname m@syu@name@\the\m@syu\endcsname}
						\xdef\m@syu@borntemp{\csname m@syu@born@\the\m@syu\endcsname}
						\xdef\m@syu@diedtemp{\csname m@syu@died@\the\m@syu\endcsname}
						\@xp\xdef\csname m@syu@name@\the\m@syu\endcsname{\csname m@syu@name@\the\m@syu@\endcsname}
						\@xp\xdef\csname m@syu@born@\the\m@syu\endcsname{\csname m@syu@born@\the\m@syu@\endcsname}
						\@xp\xdef\csname m@syu@died@\the\m@syu\endcsname{\csname m@syu@died@\the\m@syu@\endcsname}
						\@xp\xdef\csname m@syu@name@\the\m@syu@\endcsname{\m@syu@nametemp}
						\@xp\xdef\csname m@syu@born@\the\m@syu@\endcsname{\m@syu@borntemp}
						\@xp\xdef\csname m@syu@died@\the\m@syu@\endcsname{\m@syu@diedtemp}			
						\fi
						\fi
						\global\advance\m@syu\@ne
					}
					\advance\m@syu@sort@length\m@ne
					\advance\@tempcnta\@ne
				}
				%%%出力パート
				\m@syu=\@ne
				\advance\m@syu@name\@ne
				\@whilenum\m@syu<\m@syu@name
				\do{
					%%diedの調整
					\xdef\m@syu@emptychka{\csname m@syu@died@\the\m@syu\endcsname}
					\ifx\m@syu@emptychka\empty
					\@xp\def\csname m@syu@died@\the\m@syu\endcsname{\phantom{3333}}
					\else
					\@xp\c@m@syu@dieda\@xp=\csname m@syu@died@\the\m@syu\endcsname\relax
					\ifnum\c@m@syu@dieda<1000
					\let\m@syu@phantom\phantom
					\let\phantom\relax
					\def\m@syu@phantom@{\phantom{3}}
					\@xp\xdef\csname m@syu@died@\the\m@syu\endcsname{\the\c@m@syu@dieda\m@syu@phantom@}
					\let\phantom\m@syu@phantom%%%\phantom は\edefと組み合わせるとうまくいかない
					\fi
					\fi
					%%調整終わり
					\par
					\csname m@syu@name@\the\m@syu\endcsname.\hfill
					\csname m@syu@born@\the\m@syu\endcsname-\csname m@syu@died@\the\m@syu\endcsname\par
					\global\advance\m@syu\@ne}%
			}%
		}
		
		
		%---Make environment
		
		%%%\thenumXX={hoge}
		%%%\p@enumXX={\theenumXX}
		%%%\labelenumXX={(\theenumXX)}のように定義すべき. (See Source2e)
		\newcount\c@romanitemize
		\newenvironment{romanitemize}
		{\begin{enumerate}
				\@xp\def\csname labelenum\romannumeral\the\@enumdepth\endcsname
				{\@xp\c@romanitemize\@xp=\csname c@enum\romannumeral\the\@enumdepth\endcsname\relax
					(\romannumeral\the\c@romanitemize)}%
				\setlength{\parindent}{1em}%
			}{\end{enumerate}}
		
		\newenvironment{circleitemize}
		{\begin{enumerate}
				\@xp\def\csname labelenum\romannumeral\the\@enumdepth\endcsname
				{\@xp\m@syu\@xp=\csname c@enum\romannumeral\the\@enumdepth\endcsname\relax
					${\the\m@syu}^{\circ}$)}%
				\setlength{\parindent}{1em}%
			}{\end{enumerate}}
		
		\newenvironment{numitemize}
		{\begin{enumerate}
				\@xp\def\csname labelenum\romannumeral\the\@enumdepth\endcsname
				{\@xp\m@syu\@xp=\csname c@enum\romannumeral\the\@enumdepth\endcsname\relax
					$({\the\m@syu})$}%
				\setlength{\parindent}{1em}%
			}{\end{enumerate}}
		
		\newenvironment{step}
		{\begin{enumerate}
				\@xp\def\csname labelenum\romannumeral\the\@enumdepth\endcsname
				{\@xp\m@syu\@xp=\csname c@enum\romannumeral\the\@enumdepth\endcsname\relax
					Step~\the\m@syu.}%
				\setlength{\parindent}{1em}%
			}{\end{enumerate}}
		
		\newenvironment{eqv}[1][0]
		{\par
			\c@m@syu@eqv=#1\relax
			\m@syu@@=\@enumdepth
			\let\m@syu@eqv@item=\item
			\noindent\bgroup
			\ifnum \c@m@syu@eqv=\z@\relax
			\equiv@label
			\else
			\equiv@label@
			\fi}{\egroup\gdef\equiv@temp{\romannumeral}\par}
		
		\newcommand\eqvlabelset[1]{%
			\@xp\let\@xp\equiv@temp\csname equiv@label@#1\endcsname\relax
			\equiv@temp\equiv@label@roman
			\else
			\ifx\equiv@temp\equiv@label@Roman
			\else
			\ifx\equiv@temp\equiv@label@kanzi
			\else
			\ifx\equiv@temp\equiv@label@arabic
			\else
			\ifx\equiv@temp\equiv@label@Alph
			\else
			\ifx\equiv@temp\equiv@label@alph
			\else
			\@m@syu@eqvlabel
			\fi\fi\fi\fi\fi\fi
		}
		
		\newenvironment{defiterm}[2][0em]
		{\begin{enumerate}
				\@xp\def\csname labelenum\romannumeral\the\@enumdepth\endcsname
				{\@xp\m@syu\@xp=\csname c@enum\romannumeral\the\@enumdepth\endcsname\relax
					(#2\the\m@syu)}\setlength{\leftskip}{#1}}
			{\end{enumerate}}
		
		
		%%
		%%///End of defining about command which used article
		%%	
		%%
		
		%-----Test 
		
		\def\m@syu@space@char{^^`}
		
		\def\m@syu@string#1{%
			\@tfor\m@syu@member:=#1\do{%
				\ifx\m@syu@member\m@syu@space@char %%%{} も{ }も空白とみなすために必要
				\textvisiblespace
				\else
				\ifx\m@syu@member\empty
				\textvisiblespace
				\else\m@syu@member\fi
				\fi}%
		}
		
		\def\m@syu@removespace#1{%%%% #1の空白除去を\m@syu@removedspaceに格納
			\def\m@syu@removedspace{}%
			\@tfor\m@syu@member:=#1\do{%
				\ifx\m@syu@member\empty
				\edef\m@syu@removedspace{\m@syu@removedspace\m@syu@member\m@syu@space@char}%
				\else
				\edef\m@syu@removedspace{\m@syu@removedspace\m@syu@member}%
				\fi}%
		}	
		
		\newcommand{\cmd}[2][\texttt]{\eghostguarded{#1{\symbol{92}\m@syu@string{#2}}}}
		
		\newcommand{\showme}[1]{%
			\cmd{#1}%
			\par
			\m@syu@removespace{#1}%
			\@xp\ifx\csname\m@syu@removedspace\endcsname\relax
			\eghostguarded{\textbf{!undefined!}}%
			\else
			\@xp\meaning\csname\m@syu@removedspace\endcsname
			\fi
		}
		
		\def\m@syu@math{math}
		
		\newcommand{\fonttest}[2][\m@syu@math]{%
			#1#2
			\setlength{\m@syu@length}{5em}%
			\settowidth{\m@syu@length@}{#1#2}%
			\addtolength{\m@syu@length}{-\m@syu@length@}%
			\hskip\m@syu@length
			\m@syu=\@ne\relax
			\@whilenum\m@syu<27
			\do{\edef\tempa{\@Alph\m@syu}%
				\ifx #1\m@syu@math
				$\csname math#2\@xp\endcsname\@xp{\tempa}$%
				\else
				\csname #1#2\@xp\endcsname\@xp{\tempa}%
				\fi
				\global\advance\m@syu\@ne\relax}%
			\m@syu=\@ne\relax
			\@whilenum\m@syu<27
			\do{\edef\tempa{\@alph\m@syu}%
				\ifx #1\m@syu@math
				$\csname math#2\@xp\endcsname\@xp{\tempa}$%
				\else
				\csname #1#2\@xp\endcsname\@xp{\tempa}%
				\fi
				\global\advance\m@syu\@ne\relax}%
			\par
		}
		
		%----- Final
		\m@syu@elt
		
		\endinput
		%%
		%% End of file `askw3.sty'.
%</askw3>
%    \end{macrocode}
% \Finale
\endinput